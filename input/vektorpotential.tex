\section{Das Magnetische Vektorpotential}
Zur Definition eines Vektorpotentials kann nun wieder die Beziehung~\ref{eq:divfree} verwendet werden. 
\par
Wird ein Vektorpotential der Form $\exists\vec{A}\in H(curl)$
\par
\begin{equation}
	\vec{B} = \rot\vec{A}
\end{equation}
Eingef\"uhrt, so ist Gleichung~\ref{eq:max4} weiterhin erf\"ullt. Ebenso wird die De Rham Sequenz dadurch weiterhin befolgt. 
\par
\textbf{Anmerkung:} Das Vektorpotential ist also vergleichbar mit dem Skalaren Potential, welches Beispielsweise f\"ur die Elektrostatik Anwendung findet. Dieses ist definiert als:
\par
\begin{equation}
	\grad \Phi = \vec{E}
\end{equation}
Die De-Rham Sequenz ist folglich ein guter Anhaltspunkt, f\"ur die Resultierenden Vektorr\"aume nach Anwendung von Vektoroperationen.

\newpage

Grafisch kann das Vektorpotential intepretiert werden, als ein Potential, welchen stets parallel zur erzeugenden Stromdichte $\vec{J}$ des Magnetfeldes verl\"auft.
\begin{figure}[h]
	\centering
	\def\svgwidth{0.7\textwidth}
	\input{graph/potential_neu.pdf_tex}
	\caption{Grafische Darstellung des magnetischen Vektorpotentials f\"ur ein Magnetfeld, welches durch den Strom innerhalb eines Torus f\"ormigen Leiters erzeugt wird.}
	\label{fig:vecpot}
\end{figure}
\par
\textbf{Anmerkung:} Das magnetische Vektorpotential hat keine physikalische Bedeutung. Es dient lediglich als Hilfsgr\"o\ss{}e, um eine Differentialgleichung zweiter Ordnung zur L\"osung f\"ur die magnetische Flussdichte $\vec{B}$ bilden zu k\"onnen. Die gesuchte Gr\"o\ss{}e ist also am Ende weiterhin $\vec{B}$.

\subsection{Differentialgleichung zweiter Ordnung mittels Vektorpotential}
Das Vektorpotential kann anschlie\ss{}end verwendet werden, um die Differentialgleichung zweiter Ordnung herzuleiten. Dazu wird zun\"achst die Materialbeziehung f\"ur das Magnetische Feld betrachtet:
\begin{equation}
	\vec{B} = \mu\vec{H} 
\end{equation}
\par
In diese kann das Vektorpotential eingesetzt werden und man erh\"alt:
\begin{equation}
	\vec{H} = \frac{1}{\mu} \rot\vec{A}
\end{equation}
Diese Beziehung kann nun direkt in~\ref{eq:amplaw} eingesetzt werden und man Erh\"alt die Gleichung zweiter Ordnung:
\begin{align}
	\rot \vec{H} &= \vec{J} \\
	\rot \frac{1}{\mu} \rot\vec{A} &= \vec{J}
\end{align}


\newpage


\subsection{Coulomb Eichung}
\label{sec:coul}
Es gilt zu beachten, dass das Vektorpotential nicht eindeutig definiert ist. Mittels der vektoranalytischen Grundgleichungen ist leicht ersichtlich, dass eine Aufteilung des Vektorpotentials in der Form 
\par
\begin{equation}
	\vec{A} = \vec{A}^{*} + \grad\Phi
\end{equation}
Die L\"osung nicht ver\"andert, da der Gradiententerm durch die Anwendung des rot-Operators verschwindet.
\par
Folglich kann $\Phi$ in diesem Fall so gew\"ahlt werden, dass bestimmte Einschr\"ankungen erf\"ullt sind. F\"ur unseren Fall bietet sich die Coulomb Eichung an. Diese sieht vor, dass $\Phi$ so gew\"ahlt wird, dass f\"ur das Vektorpotential gilt:
\begin{equation}
	\dive \vec{A} = 0
\end{equation}
\par
Warum diese Wahl hilfreich ist wird sp\"ater erl\"autert.

