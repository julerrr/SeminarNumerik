\section{Das Magnetische Vektorpotential}
Zur Definition eines Vektorpotentials kann nun wieder die Beziehung~\ref{eq:divfree} verwendet werden. 
\par
Wird ein Vektorpotential der Form $\exists\vec{A}\in H(curl)$
\par
\begin{equation}
	\vec{A} = \rot\vec{B}
\end{equation}
Eingef\"uhrt, so ist Gleichung~\ref{eq:max4} weiterhin erf\"ullt. Ebenso wird die De Rham Sequenz dadurch weiterhin befolgt. 
\par
\textbf{Anmerkung:} Das Vektorpotential ist also vergleichbar mit dem Skalaren Potential, welches Beispielsweise f\"ur die Elektrostatik Anwendung findet. Dieses ist definiert als:
\par
\begin{equation}
	\grad \Phi = \vec{E}
\end{equation}
Die De-Rham Sequenz ist folglich ein guter Anhaltspunkt, f\"ur die Resultierenden Vektorr\"aume nach Anwendung von Vektoroperationen.

\newpage

Grafisch kann das Vektorpotential intepretiert werden, als ein Potential, welchen stets parallel zur erzeugenden Stromdichte $\vec{J}$ des Magnetfeldes verl\"auft.
\begin{figure}[h]
	\centering
	\def\svgwidth{0.7\textwidth}
	\input{graph/potential_neu.pdf_tex}
	\caption{Grafische Darstellung des magnetischen Vektorpotentials f\"ur ein Magnetfeld, welches durch den Strom innerhalb eines Torus f\"ormigen Leiters erzeugt wird.}
	\label{fig:vecpot}
	
\end{figure}

