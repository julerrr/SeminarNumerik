\section{Schwache Formulierung der Vektorpotentialgleichung}

\subsection{Lax-Milgram}
\textbf{1. Hilbertraum:}\par
Der Raum H(curl) besitzt ein inneres Produkt und ist komplett:
\begin{equation}
	(u,v)_{H(curl)} = (u,v)_{L_2} + (\rot u,\rot v)_{L_2} 
\end{equation}

\begin{equation}
	H(curl,\Omega) := \overline{\mathcal{D}(\overline{\Omega})}^{\| \cdot \|_{H(curl,\Omega)}}
\end{equation}

\textbf{2. Elliptizit\"at:}\par
\begin{equation}
	a(u,u) \geq c\|u\|^2_{H(curl)}
\end{equation}
Wir verwendet die Definition der H(curl) Norm:
\begin{equation}
	\|u\|^2_{H(curl)} = \|u\|^2_{L_2} + \|\rot u\|^2_{L_2}
\end{equation}
Mittels der Friedrichs-Typ Ungleichung f\"ur H(curl):
\begin{equation}
	\|u\|_{L_2} \leq c\|\rot u\|_{L_2}
\end{equation}
folgt nach einsetzten:
\begin{align}
	\|u\|^2_{H(curl)} &\leq C^2\|\rot u\|^2_{L_2} + \|\rot u\|^2_{L_2} \\
	\|u\|^2_{H(curl)} &\leq (1+C^2)\|\rot u\|^2_{L_2} \\
	\|\rot u\|^2_{L_2} &\geq \frac{1}{1+C^2}\|u\|^2_{H(curl)}\\
\end{align}
F\"ur $a(u,u)$ gilt:
\begin{align}
	a(u,u) &= \int_\Omega \frac{1}{\mu} \rot u\,rot u \,dx \\
	 &= \int_\Omega \frac{1}{\mu} (\rot u)^2\,dx \\
	 &\geq \inf_{\mu \in L_2} \frac{1}{\mu}\|\rot u\|^2_{L_2} \\
	 &\geq \inf_{\mu \in L_2} \frac{1}{\mu}\frac{1}{1+C^2}\|u\|^2_{H(curl)}\\
	 &= \frac{1}{\mu_1}\|u\|^2_{H(curl)}
\end{align}
Folglich ist die schwache Formulierung elliptisch mit $\frac{1}{\mu_1} = \inf_{\mu \in L_2}\frac{1}{\mu}\frac{1}{1+C^2}$.


\textbf{3. Stetig $a(\vec{A},v)$:}\par
\begin{equation}
	a(\vec{A},v) \leq C\|\vec{A}\|_{H(curl)}\|v\|_{H(curl)}
\end{equation}
Beweis mittels H\"older Ungleichung:
\begin{align}
	a(\vec{A},v) &= \int_\Omega \frac{1}{\mu} \rot\vec{A}\,\rot v\,dx \\
	 &\leq \frac{1}{\mu}\|\rot\vec{A}\|_{L_1}\|\rot v\|_{L_1} \textrm{  (Mit H\"older)} \\
	 &\leq \frac{1}{\mu}\|\rot\vec{A}\|_{L_2}\|\rot v\|_{L_2} \\
	 &\leq \frac{1}{\mu}\|\vec{A}\|_{H(curl)}\|v\|_{H(curl)} \\
\end{align}



\textbf{4. Stetig $l(v))$:}\par
\begin{equation}
	l(v) \leq \|v\|_{H(curl)}
\end{equation}
Beweis:
\begin{align}
	l(v) &= \int_\Omega \vec{J}\cdot v\,dx \textrm{   mit C-S}\\
	&\leq \|\vec{J}\|_{L_2} \|v\|_{L_2}\\
	&\leq \|\vec{J}\|_{L_2} \|v\|_{H(curl)}\\
\end{align}

Funktioniert, solange das Feld div-Frei ist. Sonst ergibt sich:
\begin{equation}
	\int_\Omega \frac{1}{\mu}\rot \vec{A} \cdot \rot v ~dx + \int_\Omega v\cdot\nabla\varphi = \int_\Omega \vec{J}\cdot v ~~~~~~~~\forall v \in H(curl)
\end{equation}