\section{Schwache Formulierung der Vektorpotentialgleichung}
Um f\"ur unsere schwache Formulierung die Randterme behandelt zu k\"onnen muss zun\"achst eine Wahl der Randbedingungen gef\"allt werden. Wir setzen $\vec{A}\times n = 0$ und $v\times n = 0$. Diese Wahl der Randwerte erf\"ullt auch unsere bereits genannte physikalische Randbedingung $B\cdot n = 0$. Das l\"asst sich \"uber die Spatproduktregel verkn\"upfen. Diese besagt:
\par
\begin{equation}
	(u \times n)\cdot v = u\cdot (n\times v)
\end{equation}
Also gilt f\"ur die physikalische Randbedingung:
\par
\begin{equation}
	\vec{B}\cdot n = \rot\vec{A}\cdot n = (\nabla\times A) \cdot n = \nabla \cdot (\vec{A}\times n) = 0
\end{equation}
Mit der Forderung $\vec{A}\times n = 0$ ist die Bedingung $\nabla \cdot (\vec{A}\times n)$ implizit erf\"ullt, und somit auch die physikalische Randbedingung.
\par
Jetzt soll die schwache Formulierung der Vektorpotentialgleichung bestimmt werden. Dazu wird wie immer die Differentialgleichung zweiter Ordnung zun\"achst \"uber das gesamte Gebiet integriert. Es ergibt sich: \par
Gesucht ist ein $v\in V = \{w \in H(curl): <v,\nabla\Phi> = 0\}$ mit $v\times n = 0$ und $\vec{A}\times n = 0$ auf $\partial\Omega$. 
\par
\begin{equation}
	\int_\Omega \rot\frac{1}{\mu}\rot \vec{A}\cdot v ~dx = \int_\Omega \vec{J}\cdot v ~dx 
\end{equation}
Mithilfe der allgemeinen partiellen Integration:
\par
\begin{equation}
	\int_\Omega \rot u\cdot v ~dx = \int_\Omega u \cdot \rot v ~dx - \int_{\partial\Omega} (u \times n)\cdot v ~ds
\end{equation}
Wenden wir dies auf unsere Differentialgleichung an, so erhalten wir:
\par
\begin{equation}
	\int_\Omega \rot\frac{1}{\mu}\rot \vec{A}\cdot v ~dx = \int_\Omega \frac{1}{\mu}\rot \vec{A} \cdot \rot v ~dx - \int_{\partial\Omega} (\frac{1}{\mu} \rot \vec{A} \times n)\cdot v ~ds = \int_\Omega \vec{J}\cdot v ~dx
\end{equation}
Um den Randterm so umzuformen, dass er zu den genannten Bedingungen passt, wird die Spatprodukt-Regel in leicht ver\"anderter Form verwendet, danach k\"onnen wir schreiben:
\par
\begin{equation}
	 -(u \times n)\cdot v = -u\cdot (n\times v) = u\cdot(v\times n) 
\end{equation}
F\"ur unsere Differentialgleichung folgt demnach:
\par
\begin{equation}
	\int_\Omega \rot\frac{1}{\mu}\rot \vec{A}\cdot v ~dx = \int_\Omega \frac{1}{\mu}\rot \vec{A} \cdot \rot v ~dx + \int_{\partial\Omega} (\frac{1}{\mu} \rot \vec{A})\cdot (v \times n) ~ds = \int_\Omega \vec{J}\cdot v ~dx
\end{equation}
Da $v\times n = 0$ auf $\partial\Omega$ lautet die schwache Fomulierung also: \par
Gesucht ist ein $v\in V = \{w \in H(curl): <v,\nabla\Phi> = 0\}$ mit $v\times n = 0$ und $\vec{A}\times n = 0$ auf $\partial\Omega$.
\par
\begin{equation}
	\int_\Omega \frac{1}{\mu}\rot \vec{A} \cdot \rot v ~dx = \int_\Omega \vec{J}\cdot v ~dx
\end{equation}
Mit der Bilinearform:
\begin{equation}
	a(\vec{A},v) =  \int_\Omega \frac{1}{\mu}\rot \vec{A} \cdot \rot v ~dx 
\end{equation}
und der Linearform:
\begin{equation}
	l(v) = \int_\Omega \vec{J}\cdot v ~dx
\end{equation}
F\"ur diese Formulierung muss folgend noch die eindeutige L\"osbarkeit gezeigt werden. Daf\"ur wird das Lax-Milgram-Lemma verwendet.

\subsection{Lax-Milgram}
F\"ur das Lax-Milgram-Lemma sind vier Eigenschaften der Formulierung zu zeigen. Diese sind:
\begin{itemize}
	\item[1] Der Raum v ist ein Hilbertraum
	\item[2] Die Bilinearform $a(\vec{A},v)$ ist elliptisch
	\item[3] Die Bilinearform $a(\vec{A},v)$ ist Stetig
	\item[4] Die Linearform $l(v)$ ist stetig
\end{itemize}
Im folgenden werden diese Bedigungen einzeln gepr\"uft.
\par
\textbf{1. Hilbertraum:}\par
Der Raum $v\in V = \{w \in H(curl): <v,\nabla\Phi> = 0\}$ mit $v\times n = 0$ besitzt ein inneres Produkt und ist komplett:
\begin{equation}
	(u,v)_{H(curl)} = (u,v)_{L_2} + (\rot u,\rot v)_{L_2} 
\end{equation}

\begin{equation}
	v := \overline{\mathcal{D}(\overline{\Omega})}^{\| \cdot \|_v}
\end{equation}

\textbf{2. Elliptizit\"at der Bilinearform $a(\vec{A},v)$:}\par
F\"ur die Elliptizit\"at der Bilinearform ist zu zeigen:
\begin{equation}
	a(u,u) \geq c\|u\|^2_{H(curl)}
\end{equation}
Wir verwenden die Definition der H(curl) Norm:
\begin{equation}
	\|u\|^2_{H(curl)} = \|u\|^2_{L_2} + \|\rot u\|^2_{L_2}
\end{equation}
Mittels der Friedrichs-Typ Ungleichung f\"ur H(curl):
\begin{equation}
	\|u\|_{L_2} \leq c\|\rot u\|_{L_2}
\end{equation}
folgt nach einsetzten:
\begin{align}
	\|u\|^2_{H(curl)} &\leq C^2\|\rot u\|^2_{L_2} + \|\rot u\|^2_{L_2} \\
	\|u\|^2_{H(curl)} &\leq (1+C^2)\|\rot u\|^2_{L_2} \\
	\|\rot u\|^2_{L_2} &\geq \frac{1}{1+C^2}\|u\|^2_{H(curl)}\\
\end{align}
F\"ur $a(u,u)$ gilt:
\begin{align}
	a(u,u) &= \int_\Omega \frac{1}{\mu} \rot u\,rot u \,dx \\
	 &= \int_\Omega \frac{1}{\mu} (\rot u)^2\,dx \\
	 &\geq \inf_{\mu \in L_2} \frac{1}{\mu}\|\rot u\|^2_{L_2} \\
	 &\geq \inf_{\mu \in L_2} \frac{1}{\mu}\frac{1}{1+C^2}\|u\|^2_{H(curl)}\\
	 &= \frac{1}{\mu_1}\|u\|^2_{H(curl)}
\end{align}
Folglich ist die schwache Formulierung elliptisch mit $\frac{1}{\mu_1} = \inf_{\mu \in L_2}\frac{1}{\mu}\frac{1}{1+C^2}$.


\newpage


\textbf{3. Stetigkeit der Bilinearform $a(\vec{A},v)$:}\par
F\"ur die Stetigkeit der Bilinearform ist zu zeigen:
\begin{equation}
	a(\vec{A},v) \leq C\|\vec{A}\|_{H(curl)}\|v\|_{H(curl)}
\end{equation}
Beweis mittels Cauchy-Schwarz, sowie der H\"olderschen Ungleichung:
\begin{align}
	a(\vec{A},v) &= \int_\Omega \frac{1}{\mu} \rot\vec{A}\,\rot v\,dx \\
	 &\leq \frac{1}{\mu}\|\rot\vec{A}\|_{L_1}\|\rot v\|_{L_1} ~~~~~~\textrm{(mit H\"older)} \\
	 &\leq \frac{1}{\mu}\|\rot\vec{A}\|_{L_2}\|\rot v\|_{L_2} \\
	 &\leq \frac{1}{\mu}\|\vec{A}\|_{H(curl)}\|v\|_{H(curl)} \\
\end{align}



\textbf{4. Stetigkeit der Linearform $l(v))$:}\par
F\"ur die Stetigkeit der Linearform ist zu zeigen:
\begin{equation}
	l(v) \leq \|v\|_{H(curl)}
\end{equation}
Beweis:
\par
\begin{align}
	l(v) &= \int_\Omega \vec{J}\cdot v\,dx \textrm{   mit C-S}\\
	&\leq \|\vec{J}\|_{L_2} \|v\|_{L_2}\\
	&\leq \|\vec{J}\|_{L_2} \|v\|_{H(curl)}\\
\end{align}
Also ist Lax-Milgram f\"ur die Formulierung erf\"ullt und das Problem besitzt eine Eindeutige L\"osung.
