\newpage
\section{H(curl) R\"aume}
F\"ur unsere Funktionen m\"ussen nun noch die bereits in der De-Rham Sequenz eingef\"uhrten Funktionenr\"aume definiert werden. F\"ur das Vektorpotential wird der Vektorraum H(curl) verwendet. Dieser ist gegeben als:
\begin{equation}
	H(curl) = u\in [L_2]^3:\rot u\in [L_2]^3
	\label{eq:hcurl}
\end{equation}
\par
Im H(curl) Raum ist die innere Norm gegeben als:
\begin{equation}
	\|u\|_{H(curl)} = \sqrt{\|u\|^2_{L_2} + |u|^2_{H(curl)}}
\end{equation}
bzw.
\begin{equation}
	\|u\|_{H(curl)} = \sqrt{\|u\|^2_{L_2} + \|\rot u\|^2_{L_2}}
\end{equation}