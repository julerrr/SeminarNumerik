\section{Die Helmholtz Zerlegung}
Für eine Vektorfunktion $u\in L_2$ ist eine Zerlegung in eine Gradientenfunktion, sowie eine Rotationsfunktion m\"oglich~\cite{Schberl2009NumericalMF}.
\begin{equation}
	u = \nabla\Phi + \rot\psi
\end{equation}
\par
Dabei wird $\Phi$ als Skalarpotential, und $\psi$ als Vektorpotential bezeichnet. Wie bereits unter Abschnitt~\ref{sec:coul} beschrieben, ist das Vektorpotential nicht eindeutig definiert, da dies selbst wieder in der Form
\par
\begin{equation}
	\psi = \psi^{*} + \nabla\chi 
\end{equation}
gesplittet werden kann. Dadurch kann auch hier eine bestimmte Einschr\"ankung gew\"ahlt werden, wie die Coulomb Eichung ($\dive \psi = 0$).