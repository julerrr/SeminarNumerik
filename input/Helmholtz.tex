\section{Die Helmholtz Zerlegung}
Für eine Vektorfunktion $u\in L_2$ ist eine Zerlegung in eine Gradientenfunktion, sowie eine Rotationsfunktion m\"oglich~\cite{Schberl2009NumericalMF}.
\begin{equation}
	u = \nabla\Phi + \rot\psi
\end{equation}
\par
Dabei wird $\Phi$ als Skalarpotential, und $\psi$ als Vektorpotential bezeichnet. Wie bereits unter Abschnitt~\ref{sec:coul} beschrieben, ist das Vektorpotential nicht eindeutig definiert, da dies selbst wieder in der Form
\par
\begin{equation}
	\psi = \psi^{*} + \nabla\chi 
\end{equation}
gesplittet werden kann. Dadurch kann auch hier eine bestimmte Einschr\"ankung gew\"ahlt werden, wie die Coulomb Eichung ($\dive \psi = 0$).

\subsection{Sonderfall f\"ur $u\in H(curl)$}
F\"ur den Fall, dass die Vektorfunktion u im H(curl) Raum liegt, kann eine Zerlegung in der Form \par
\begin{equation}
	u = \nabla\Phi + z
\end{equation}
wobei $Z = \rot \psi$ ist, gefunden werden.
Dabei gilt: $\Phi\in H^1$ und $z\in [H^1]^3$\par
Au\ss{}erdem gilt hierbei, dass ein $q\in H(div)$ existiert, sodass:\par
\begin{equation}
	\rot z = q = \rot u
	\label{eq:qdef}
\end{equation}
Diese Beziehung folgt erneut aus der Gleichung der Vektoranalysis, nach der der Gradiententerm durch Anwendung einer Rotation verschwindet.
\par
Weniger trivial ist die Beziehung der Funktionenr\"aume. Hierzu werden zun\"achst Einschr\"ankungen eingef\"uhrt:
\begin{align}
	\dive z &= 0\\
	z\cdot n &= 0
\end{align}
In (\cite{Schberl2009NumericalMF}, S.24ff) Wird der Beweis gef\"uhrt, indem die Funktion q mit null auf ganz $\Re^3$ erweitert wird. \"Uber die Fouriertranformation von q (\ref{eq:four0} - \ref{eq:four}) kann daraufhin die $H^1$ Seminorm von z unter der Einschr\"ankung $\dive z =0$ in ganz $\Re^3$ bestimmt werden.
\par
\begin{align}
	\mathcal{F}(\nabla q) &= 2\pi i\xi\mathcal{F}q \label{eq:four0}\\
	\mathcal{F}(\rot q) &= 2\pi i\xi\times\mathcal{F}q \\
	\mathcal{F}(\dive q) &= 2\pi i\xi\cdot\mathcal{F}q
	\label{eq:four}
\end{align}
\par
\begin{equation}
	\|\nabla z\|_{L_2(\Re^3)} = \|2\pi i \xi \tilde{z}\|_{L_2(\Re^3)} = \bigg{\|}\frac{\xi\xi\times\tilde{q}}{|\xi|^2}\bigg{\|} = \|\tilde{q}\|_{L_2(\Re^3)} = \|q\|_{L_2(\Omega)}
\end{equation}

Da q nur erweitert wurde, l\"asst sich aus der Allgemeinen Beziehung daraufhin wieder die Einschr\"ankung t\"atigen, sodass die Beziehung auf jedem Teilgebiet $\Omega\in \Re^3$ g\"ultig ist.