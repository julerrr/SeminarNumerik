\section{Gemischte Formulierung}
Wir entfernen nun die Bedingung der Divergenzfreiheit aus dem Raum v, also: $v\in V = \{w \in H(curl)\}$ mit $v\times n = 0$ und $\vec{A}\times n = 0$ auf $\partial\Omega$. Das Lax-Milgram-Lemma ist f\"ur nun nicht mehr erf\"ullt und es ergibt sich: 
\par
\begin{equation}
	\int_\Omega \frac{1}{\mu}\rot \vec{A} \cdot \rot v ~dx + \int_\Omega v\cdot\nabla\varphi = \int_\Omega \vec{J}\cdot v ~~~~~~~~\forall v \in H(curl)
\end{equation}
Um dieses Problem trotzdem eindeutig l\"osen zu k\"onnen, wird die Bedingung der Divergenzfreiheit als zus\"atzliche Einschr\"ankung eingef\"uhrt. Man erh\"alt ein Sattelpunktsproblem der Form:
\par
\begin{align}
	\int_\Omega \frac{1}{\mu}\rot \vec{A} \cdot \rot v ~dx + \int_\Omega v\cdot\nabla\varphi & = \int_\Omega \vec{J}\cdot v &\forall v \in H(curl) \\
	\int_\Omega \vec{A}\cdot \nabla\psi &= 0 &\forall\psi\in H^1
\end{align}


\subsection{Inf-Sup-Bedingung nach Brezzi}

