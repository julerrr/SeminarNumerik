\section{Gemischte Formulierung}
Wir entfernen nun die Bedingung der Divergenzfreiheit aus dem Raum v, also: $v\in V = \{w \in H(curl)\}$ mit $v\times n = 0$ und $\vec{A}\times n = 0$ auf $\partial\Omega$. Das Lax-Milgram-Lemma ist f\"ur nun nicht mehr erf\"ullt und es ergibt sich: 
\par
\begin{equation}
	\int_\Omega \frac{1}{\mu}\rot \vec{A} \cdot \rot v ~dx + \int_\Omega v\cdot\nabla\varphi = \int_\Omega \vec{J}\cdot v ~~~~~~~~\forall v \in V
\end{equation}

\newpage 

Um dieses Problem trotzdem eindeutig l\"osen zu k\"onnen, wird die Bedingung der Divergenzfreiheit als zus\"atzliche Einschr\"ankung eingef\"uhrt. Man erh\"alt ein Sattelpunktsproblem der Form:
\par
\begin{align}
	\int_\Omega \frac{1}{\mu}\rot \vec{A} \cdot \rot v ~dx + \int_\Omega v\cdot\nabla\varphi & = \int_\Omega \vec{J}\cdot v &\forall v \in V \\
	\int_\Omega \vec{A}\cdot \nabla\psi &= 0 &\forall\psi\in H^1_0
\end{align}
Dabei sind:
\begin{align}
	a(\vec{A},v) &= \int_\Omega \frac{1}{\mu}\rot \vec{A} \cdot \rot v ~dx \\
	b(v,\nabla\varphi) &=  \int_\Omega v\cdot\nabla\varphi \\
	l(v) &= \int_\Omega \vec{J}\cdot v 
\end{align}

\par
Um die Wohlgestelltheit dieses Problems festzustellen sind zwei Gegebenheiten zu pr\"ufen. Die erste Bedingung ist die Kernelliptizit\"at des Terms $a(\vec{A},v)$. Dies wurde bereits f\"ur das Lax-Milgram Lemma gezeigt, somit ist diese Bedigung erf\"ullt. Die zweite Bedingung ist die Einhaltung der Inf-Sup-Bedingung (LBB-Bedingung). Diese wird im folgenden gezeigt.


\subsection{Inf-Sup-Bedingung nach Brezzi}
Wir schreibem die die Inf-Sup-Bedingung f\"ur die Bilinearform $b(v,\nabla\varphi)$:
\par
\begin{equation}
	\sup_{v\in V} \frac{\int v\cdot\nabla\varphi}{\|v\|_{V}} \geq \beta_1\|\nabla\varphi\|_{L_2}
\end{equation}
Zum Beweis w\"ahlen wir $v=\nabla\varphi$, und erhalten damit direkt:
\par
\begin{align}
	\sup_{v\in V} \frac{\int v\cdot\nabla\varphi}{\|v\|_V}
	&\geq \frac{\int \nabla\varphi\cdot\nabla\varphi}{\|\nabla\varphi\|_V} \\
	&= \frac{\int (\nabla\varphi)^2}{\|\nabla\varphi\|_{V}}\\
	&= \frac{\|\nabla\varphi\|^2_{L_2}}{\|\nabla\varphi\|_{V}} \\
	&= \frac{\|\nabla\varphi\|^2_{L_2}}{\sqrt{\|\nabla\varphi\|^2_{L_2} + \|\rot{\nabla\varphi}\|^2_{L_2}}} \\
	&= \frac{\|\nabla\varphi\|^2_{L_2}}{\|\nabla\varphi\|_{L_2}} \\
	&= \|\nabla\varphi\|_{L_2}
\end{align}
