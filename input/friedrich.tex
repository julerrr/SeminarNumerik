\newpage
\section{Friedrichs Ungleichung f\"ur H(curl) Funktionenr\"aume}
Unter Verwendung der Helmholtz Zerlegung:
\par
\begin{equation}
	u = \nabla \Phi + z
\end{equation}
mit $z = \rot\psi$
\par
Sowie der Forderung (aus $\dive u = 0$):
\par
\begin{equation}
	(u,\nabla\psi) = 0 ~~~~~~\forall \psi\in H^1_0
\end{equation}
Ebenso ist aus der De-Rham Sequenz bekannt, dass $\dive z = 0$ ergeben muss. Folglich l\"asst sich schreiben:
\par
\begin{align}
	(z, \nabla\psi) &= 0\\
	(z, \nabla\psi) &= (u - \nabla\Phi, \nabla\psi) = 0  
\end{align}
Nun l\"asst sich \"uber die Absch\"atzung~\cite[Lemma 4.2]{hiptmair2002finite} auf einem einfachen und beschr\"ankten Lipschitz-Gebiet, die $L_2$ Norm durch die Norm der Rotation, der Divergenz und der Randbedingung mit den gezeigten Beziehungen $\dive u = 0$ und $n\times u = 0$ schreiben, dass:
\par
\begin{equation}
	\|u\|_{L_2} \leq c[\|\rot u\|_{L_2} + \|\dive u\|_{L_2} + |n\times u|] = c\|\rot u\|_{L_2}
\end{equation}
Also haben wir die Friedrichs-Typ Ungleichung f\"ur H(curl):
\begin{equation}
	\|u\|_{L_2} = c\|\rot u\|_{L_2}
\end{equation}
